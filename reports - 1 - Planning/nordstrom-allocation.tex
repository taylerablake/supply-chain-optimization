\documentclass[11pt, oneside]{article}   	% use "amsart" instead of "article" for AMSLaTeX format
\usepackage{geometry}                		% See geometry.pdf to learn the layout options. There are lots.
\geometry{letterpaper}                   		% ... or a4paper or a5paper or ... 
%\geometry{landscape}                		% Activate for rotated page geometry
%\usepackage[parfill]{parskip}    		% Activate to begin paragraphs with an empty line rather than an indent
\usepackage{graphicx}				% Use pdf, png, jpg, or eps§ with pdflatex; use eps in DVI mode
								% TeX will automatically convert eps --> pdf in pdflatex		
\usepackage{amssymb}
\usepackage{amsmath}

%SetFonts

%SetFonts


\title{Optimal Inventory Allocation for E-commerce Fulfillment}
\author{Christopher Holloman \\ Chief Data Scientist \\ Advanced Analytics Practice \\ Information Control Company}
\date{Last updated: \today}							% Activate to display a given date or no date

\begin{document}
\maketitle

Over the past several years, the importance of e-commerce to traditional retailers has grown dramatically.  While shoppers still purchase goods from brick-and-mortar stores regularly, Amazon and other on-line retailers have taken a significant amount of business from traditional stores.  All retailers must serve customers online to survive, and it's important that they are able to meet the same standards for delivery speed as online-only retailers.

One of the key aspects of meeting these high standards is to optimally allocate inventory to fulfillment centers to ensure that goods can be delivered to clients' home addresses quickly.  Optimal allocation must account for two primary factors:

\begin{itemize}
\item Cost of delivering units from fulfillment centers to delivery addresses
\item Cost of transferring units between fulfillment centers to respond to shifts in demand
\end{itemize}

We present a strategy for allocating units for a single SKU to an arbitrary number of fulfillment centers.

\section{Assumptions}

There is a long process from determining which styles a retailer will sell to determining exactly how those styles are delivered to consumers.  Among the problems to be solved along the way are the following:

\begin{itemize}
\item How to forecast the number of units to be sold
\item How to forecast the fraction of units sold in different regions
\item How to quantify the uncertainty about the number of units to be sold
\item How to determine the number of units to allocate to each store
\item How to determine the number of units to store in each distribution center
\item How to determine the number of units to place in each e-commerce fulfillment center
\item How to determine which e-commerce fulfillment center fills each order
\item How to determine when units are shipped from one e-commerce fulfillment center to another
\end{itemize}

For the purpose of this paper, we assume that the retailer has already developed a statistical model forecasting the number of units that will be purchased through the online channel in each of several regions.  We also assume the retailer has quantified their uncertainty in those forecasts.  Further, we assume that the retailer has a set of rules specifying the logic for fulfilling orders from fulfillment centers and for transferring units between fulfillment centers.

\section{Forecasting Online Sales}

%\subsection{}



\end{document}  